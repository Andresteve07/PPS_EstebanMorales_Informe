% Chapter Template

\chapter{Conclusiones} % Main chapter title

\label{Chapter6} % Change X to a consecutive number; for referencing this chapter elsewhere, use \ref{ChapterX}

\lhead{Chapter 6. \emph{Conclusiones}} % Change X to a consecutive number; this is for the header on each page - perhaps a shortened title

%----------------------------------------------------------------------------------------
%	SECTION 1
%----------------------------------------------------------------------------------------
Durante el desarrollo de la presente práctica se entendió las ventajas de implementar un sistema de software utilizando un patrón de arquitectura. La inspección de código escrito por profesionales del área introdujo conceptos de programación Java avanzados tales como el uso de clases anónimas y la implementación de clases y métodos genéricos. Se pudo observar las técnicas empleadas para realizar las pruebas sobre el código implementado y fue necesario invertir tiempo en la investigación de las librerías y framework para pruebas tales como Mockito y Espress.\\
De manera indirecta se pudo observar cómo la definición de una arquitectura de estas características no solo tiene impacto en la organización y testabilidad del código sino también introduce un procedimiento de trabajo tanto para la adición de nuevas funcionalidades sino para la remoción de errores y la inspección del código en general.\\
Como una desventaja notoria se menciona la empinada curva de aprendizaje para la inclusión de nuevos miembros en un hipotético equipo de desarrollo. Así mismo se hizo evidente que todos los conceptos de abstracción que fueron introducidos se traducen en un aumento notable en la cantidad de lineas de código meramente dedicadas a mantener la estructura del diseño pero que no proveen una funcionalidad concreta al sistema.\\
Finalmente, se observaron inconsistencias entre el planteo teórico de la arquitectura y la implementación real del software mayormente por la dificultad técnica y concesiones en la que incurrieron los programadores para disminuir la verbosidad de algunos componentes o interacciones.
