% Chapter Template

\chapter{Conclusiones} % Main chapter title

\label{Chapter7} % Change X to a consecutive number; for referencing this chapter elsewhere, use \ref{ChapterX}

\lhead{Chapter 7. \emph{Conclusiones}} % Change X to a consecutive number; this is for the header on each page - perhaps a shortened title

%----------------------------------------------------------------------------------------
%	SECTION 1
%----------------------------------------------------------------------------------------
En el presente proyecto se logró diseñar y desarrollar un prototipo funcional de un sistema de computación completo, el cual permitió integrar y profundizar muchos de los conocimientos adquiridos en la carrera de Ingeniería en Computación.\\
Por un lado,partiendo de la Ingeniería de Software como eje de desarrollo, se utilizaron distintas metodologías y técnicas que permitieron realizar un sistema de escalable, verificable y mantenible. El proyecto presentó el desafío de poder comunicar sistemas programados en distintos lenguajes con distintos paradigmas, por lo que se debió realizar un análisis exhaustivo de las posibles interfaces entre ellos y sus posibles limitaciones. \\
Por otro lado,hubo que investigar y elegir la opción de arquitectura de \textit{Hardware}, más conveniente en cada caso, no solo por sus capacicades técnicas, sino también por su factibilidad económica y política (restricciones de disponibilidad de microcontroladores). Además, se realizó un prototipo de un circuito digital impreso utilizando un software de diseño asistido, que fué adaptado al microcontrolador existente para el desarrollo del sistema embebido.\\
Se estudiaron distintos mecanismos y protocolos de comunicación, para poder lograr la interacción de los sistemas entre sí, tanto de manera local(LAN), como por Internet. Esto presento varios inconvientes en la etapas de pruebas de integración debido a las limitaciones de que imponían los proveedores de Internet, pero que fueron resueltas a través de modificaciones en el dispositivo de \textit{ruteo}.\\
Se realizaron varios prototipos con distintos niveles de abstracción que fueron validándose con el usuario y que permitieron mejorar en cada etapa el sistema final. Debido a que el sistema se realizó de forma modular y concurrente entre sus sub-sistemas, las modificaciones que el usuario sugirió en cada caso no tuvieron un impacto considerable en cuestiones de tiempo y desarrollo, gracias a esta metodología de desarrollo.\\
Por último, cabe destacar que se cumplieron con todos los objetivos planteados en un principio, y esto fué debido a la planificación y diseño previo del sistema. 


\newpage

\section{Trabajos futuros}

A continuación se describen algunas de las posibles mejoras al sistema que podrían realizarse con el fin de fortalecer el sistema tanto en funcionalidad como en fiablilidad y seguridad:

\begin{itemize}
\item Alarma:
	\begin{itemize}
	\item DHCP y mDNS para búsqueda automática del mismo.
	\item JSON como formato de mensajes.
	\item Algoritmos de encriptación y \textbf{hash} para mensajes y autenticación.
	\end{itemize}
\item Aplicación:
	\begin{itemize}
	\item Recepción de notificaciones \textit{push}.
	\item Configuración de usuarios.
	\item Búsqueda automática de alarma en la red local.
	\end{itemize}
\item Servidor:
	\begin{itemize}
	\item Utilización de protocolos de comunicación tipo REST ó Websockets para unificar la conexión con alarma y aplicación.
	\item Jerarquía de operadores de sistema.
	\end{itemize}
\item Diseño general de sistema:
	\begin{itemize}
	\item Utilización de \textit{Software} de Integración Contínua.
	\item \textit{SysUML} como software para la especificación de requerimientos y diseño de sistema basado en UML.
	\end{itemize}
\end{itemize}

