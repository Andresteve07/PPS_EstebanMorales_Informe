% Chapter Template

\chapter{Objetivos y criterios de la metodología usada} % Main chapter title

\label{Chapter2} % Change X to a consecutive number; for referencing this chapter elsewhere, use \ref{ChapterX}

\lhead{Capítulo 2. \emph{Objetivos y criterios de la metodología usada}} % Change X to a consecutive number; this is for the header on each page - perhaps a shortened title

%----------------------------------------------------------------------------------------
%	SECTION 1
%----------------------------------------------------------------------------------------

\section{Objetivos}

Se requiere desarrollar un sistema que permita monitorizar alarmas a través de Internet.

%-----------------------------------
%	SUBSECTION 1
%-----------------------------------

\section{Limitaciones}

A continuación, se explicitan las limitaciones del sistema a desarrollar, impuestas por el director del proyecto:
\begin{itemize}
\item Sensores a monitorizar de tipo ON-OFF.
\item Máximo 8 sensores por alarma.
\item Conexiones simultáneas limitadas a monitorizar.
\item Envío de alertas por red sin encriptación.
\end{itemize}

%----------------------------------------------------------------------------------------
%	SECTION 2
%----------------------------------------------------------------------------------------
\clearpage

\section{Criterios de la Metodología de diseño de Hardware}

En la siguiente sección se presentan y analizan las distintas opciones de \textit{hardware} disponibles en el mercado para el desarrollo del sistema. Éste se divide en 3 subsistemas y se presentan de manera separada.
\clearpage

\subsection{Subsistema Alarma}

En este caso existen varias alternativas, que se presentan a continuación:

\begin{table}[ht]
  	\begin{center}
  	\hspace*{-2cm}
	  \begin{tabular}{|l|l|l|l|l|}
		\hline      
        Marca				& Placa de desarrollo & Características		& Ventajas		& Desventajas				\\
		\hline \hline
    	PIC					&16F877A			& Arq:RISC					&+Bajo costo	&-Tecnología antigua		\\
    						&					& Bus 8 bits				&+Soporte		&-Sin conectividad	 		\\
    						&					& Frec:20 MHz 				&+Bajo consumo	&de red propia				\\
    						&					& RAM:368 B 				&				&-Baja \textit{Performance}	\\
							&					& 8KB Flash 				&				&-No SO						\\
							&					& 256 B EEPROM 				&				& Programador externo	 	\\		
    						&					& I2C,SPI,USART				&				&		\\ 
    						&					& ADC 10 bits(6)			&				&		\\    					
    \hline
    ATmel					&Arduino UNO		& Arq:RISC					&+Soporte		&-Recursos limitados		\\
    						&(ATmega 328)		& Bus 8 bits				&+Software		&-Conectividad de red	 		\\
    						&					& Frec:16 MHz 				&+\textit{Hardware} 		&-Baja \textit{Performance}	 	\\
    						&					& RAM: 1KB 					&complementario				&-No SO							\\
							&					& 8KB Flash 				&+Bajo consumo				&							\\
							&					& 512 B EPROM 				&				&		\\
    						&					& I2C,SPI,USART				&				&		\\ 
    						&					& ADC 10 bits(5)			&				&		\\    		
    \hline
    ARM 					&K64F				& Arq:RISC					&+Alta \textit{Performance}		&-Relativo alto		\\
    						&(Cortex M4)		& Bus: 32 bits	 			&+Múltiples \textit{RTOSs}		& consumo		 	\\
    						&					& Frec:120 Mhz 				&+Ethernet						&-Alto costo		\\
    						&					& RAM: 256KB 				&+\textit{Debug in}				&					\\
							&					& Flash: 1 MB  				& \textit{circuit}				&					\\
							&					& 512 B EEPROM 				&				&			\\
    						&					& Ethernet					&				&		\\
    						&					& SDHC						&				&		\\
    					    &					& USB con OTG				&				&		\\
    						&					& I2C,SPI,USART				&				&		\\ 
    						&					& PWM						&				&		\\
    \hline
   Intel-x86  				&Intel Galileo 		& Arq:CISC					&+Alta \textit{Performance}		&-Alto consumo		\\
       						&(2da Gen)			& Bus: 32 bits				&+Software						&-Conectividad de red	 		\\
    						&					& Frec: 400 MHz				&+Múltiples \textit{RTOSs}		&-Alto costo	 	\\
    						&					& RAM: 256 MB				&+Ethernet						&-Disponibilidad		\\
							&					& Flash: 8 Mb 				&				&		\\
							&					& EEPROM: 8 Kb 				&				&		\\
    						&					& Ethernet			&				&		\\
    						&					& SDHC				&				&		\\
    					    &					& USB con OTG		&				&		\\
    						&					& I2C,SPI,USART		&				&		\\ 
    						&					& PWM				&				&		\\
    \hline
    
  \end{tabular}
  \hspace*{-2cm}
	\caption{Tabla de Arquitecturas de Hardware.}
	\label{Tabla de Arquitecturas de Hardware.}
	\end{center}
\end{table}
\clearpage

Se decide por la utilización de la Arquitectura \textit{ARM}, en particular, del microcontrolador de la serie \textit{Cortex}, familia M4, con la placa de desarrollo \textit{Freescale} K64F.\cite{embedded2} \\
La misma ofrece prestaciones superiores a las demás en \textit{performance}, como en periféricos necesarios para el desarrollo tales como interfaz \textit{Ethernet},\textit{ Debug in Circuit} y la posibilidad de utilizar Sistemas Operativos de tiempo real.\\



\subsection{Subsistema Servidor}

Se opta por utilizar un servidor con arquitectura x86, ya que es la más difundida.

\subsection{Subsistema Aplicación}
Debido al requerimiento de la aplicación para \textit{smartphone}, en un principio sobre SO \textit{Android}, el cual corre casi en la totalidad de equipos con arquitectura \textit{ARM}, se elige esta arquitectura.

\newpage
\section{Criterios de la Metodología de Desarrollo de Software}
Se detalla en un principio la metodología de desarrollo y el modelado para el sistema general y posteriormente el lenguaje de programación elegido en cada caso.\citep{softIng2Book},\cite{softIng3Book},\cite{softIng5Book}\\
Por último, se describen los SOs utilizados en cada caso.\\

	\subsection{Modelo y Metodología de desarrollo} 
Existen distintos tipos de modelos y de metodologías de desarrollos de software, los cuales se plantean en las tablas \ref{Tabla de Modelos de desarrollo de Software.} y \ref{Tabla de Metodologias de desarrollo de Software.} correspondientemente, explicitando sus características y ventajas/desventajas con respecto a este proyecto:\\

\begin{table}[ht]
  	\begin{center}
  	\hspace*{-3cm}
	  \begin{tabular}{|l|l|l|l|}
		\hline      
        Modelo			& Características										& Ventajas						& Desventajas			\\
		\hline \hline
    	Cascada			& -Primer modelo y base de todos los demás.			&+Fácil de implementar. 		&-Inflexible.			\\			
    					& -Orden riguroso de las etapas.						&+Modelo conocido y   			&-Lento.					\\	
    					&														&utilizado con frecuencia.		&-Impredecible.			\\
    					& 														&+Orientado a documentos.  		&							\\			
    	\hline
    	Espiral			& -Las actividades se conforman en una  				&+El análisis del riesgo se 	&-Lento				\\
    					& espiral, en la que cada iteración 					&hace de forma explícita. 		&-Costoso				\\
    					& representa un conjunto de actividades.				&+Objetivos de calidad			&-Requiere experiencia\\
    					& -Se basa en el riesgo que aparece a 					&+Integra el desarrollo 		& en identificación	\\    
    					& la hora de desarrollar software.						&con el mantenimiento			& de riesgos.						\\	
    	\hline
    	Tipo V			& -Representa la secuencia de desarrollo 				&+Minimización de los  			&-Inflexible.			\\ 
    					&  del ciclo de vida de un proyecto.					&riesgos del proyecto.			&							\\ 
    					& -Se describen las actividades y resultados 			&+Garantiza la Calidad.			&							\\ 
    					& que deben producirse durante el desarrollo.			&+El esfuerzo para el 			&							\\	
    					& -El lado izquierdo de la V representa la 			&puede ser	calculado,			&							\\	
    					& descomposición de las necesidades, y la  			&estimado y controlado.  		&							\\	
    					& creación de las especificaciones del sistema. 		&								&							\\	
    					& -El lado derecho de la V representa la 				&  								&							\\	
    					& integración de las piezas y su verificación. 		&  								&						\\	
    					& -V significa Verificación y validación.				&  								&					\\	
    	\hline
    	Iterativo		& -Creado en respuesta a las debilidades del			&+Los usuarios pueden utilizar &-Requiere de un	\\
    	e				&modelo tradicional de cascada. 						&los incrementos iniciales		&cliente involucrado  	\\	
    	Incremental		& -Conjunto de tareas agrupadas en pequeñas  			&como prototipos.  				&durante todo \\	
    					&etapas repetitivas (iteraciones).						&+Pocas probabilidades de 		&el proyecto.			\\
						& -En cada iteración los componentes logran 			&riesgo en el sistema.			&Iteraciones costosas    	\\
  						&evolucionar el producto dependiendo de los			&+Permite separar la			&    	\\    
    					& completados de las iteraciones antecesoras,			&complejidad del proyecto.		&    	\\
    					& agregando más opciones de requisitos.  				&+con los usuarios.				&    	\\
    	\hline
    	\end{tabular}
  		\hspace*{-4cm}
	\caption{Tabla de Modelos de desarrollo de Software.}
	\label{Tabla de Modelos de desarrollo de Software.}
	\end{center}
\end{table}

\newpage

